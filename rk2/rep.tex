\RequirePackage{shellesc}
\documentclass[a4paper]{article}
\usepackage[document]{ragged2e}
\usepackage{fontspec}
\usepackage{geometry}
\usepackage{titlesec}
\usepackage{graphicx}
\usepackage{float}
\usepackage{minted}
\usepackage{enumitem}
\usepackage{indentfirst}
\usepackage[Export]{adjustbox} % Used to constrain images to a maximum size
\usepackage[font=large]{caption}
\geometry{left=2cm}
\geometry{right=2cm}
\geometry{top=2cm}
\geometry{bottom=2cm}
\setmainfont{Liberation Serif}
\setmonofont[Scale=0.7]{Hack}
\titlelabel{\thetitle. }
\renewcommand{\tablename}{Таблица}
\renewcommand{\figurename}{Рисунок}
\counterwithin{table}{section}
\counterwithin{figure}{section}
\captionsetup{justification=raggedright,singlelinecheck=false}
\renewcommand\large{\fontsize{14}{16}\selectfont}
\renewcommand{\theFancyVerbLine}{\textbf{\arabic{FancyVerbLine}}}
\sloppy

\setminted{
  linenos,
  breaklines,
  breakanywhere,
  breaksymbolleft=,
  fontsize=\fontsize{12}{12}\selectfont,
  numbersep=5pt,
}
\usemintedstyle{emacs}

\graphicspath{ {../4/pics/} }

\begin{document}
  \fontsize{14}{16}\selectfont

  \begin{titlepage}
    \begin{minipage}{0.2\textwidth}
      \includegraphics[scale=0.4]{logo}
    \end{minipage}
    \begin{minipage}{0.7\textwidth}\centering
      \fontsize{10}{12}\selectfont
      \textbf{
        Федеральное государственное бюджетное образовательное учреждение \\
        высшего профессионального образования \\
        «Московский государственный технический университет имени Н.Э. Баумана» \\
        (МГТУ им. Н.Э. Баумана)
      }
    \end{minipage}

    \vspace{5cm}
    \centering
    \fontsize{16}{20}\textbf{
      Рубежный контроль №2 \\
      по курсу «Методы машинного обучения» \\
    }

    \vspace{5cm}
    \begin{flushright}
    Выполнил \\
    студент группы ИУ5-22М \\
    XXXX \\
    \end{flushright}
    \vspace*{\fill}
    Москва, 2023
  \end{titlepage}

  \justifying
  \setlength{\parindent}{1.25cm}

  \section{Задание}
  Для одного из алгоритмов временных различий, реализованных Вами в соответствующей лабораторная работе:
  \begin{itemize}
    \item \textbf{SARSA}
    \item Q-обучение
    \item Двойное Q-обучение
  \end{itemize}
  осуществите подбор гиперпараметров. Критерием оптимизации должна являться суммарная награда.

  \section{Текст программы}
  \inputminted{python}{rk2.py}
  Программа производит автоматический перебор параметров и находит оптимальные.
  \pagebreak

  \section{Графики}
  Параметр eps:
  \begin{figure}[H]
    \includegraphics[scale=0.5]{rk21}
  \end{figure}
  Параметр lr:
  \begin{figure}[H]
    \includegraphics[scale=0.5]{rk22}
  \end{figure}
  \pagebreak
  Параметр gamma:
  \begin{figure}[H]
    \includegraphics[scale=0.5]{rk23}
  \end{figure}
  Лучшие параметры: eps=0.1, lr=0.39, gamma=0.93
  \begin{figure}[H]
    \includegraphics[scale=0.5]{rk24}
  \end{figure}
  \pagebreak
\end{document}
